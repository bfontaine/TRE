%!TEX root = rapport.tex

\section{\label{sec:terminologie}Terminologie}

Pour mieux comprendre la suite de ce document, il est nécessaire d’introduire
le vocabulaire utilisé dans le cadre de ce type d’étude.

\subsection{Facebook}

\fb{} est un réseau social en ligne créé en 2004 qui compte plusieurs
centaines de millions d’utilisateurs inscrits. Chaque utilisateur dispose d’un
profil avec un \q{mur} qui représente la suite de toutes ses publications,
qui peuvent être textuelles, mais également des photos, vidéos, ou liens
internes et externes à la plateforme. Les utilisateurs sont liés entre eux par
une relation symétrique d’\q{amitié}, où l’un des deux acteurs de cette
relation doit envoyer une requête à l’autre acteur qui doit l’accepter pour que
la relation soit créée.\footnote{Il est bien entendu possible de refuser une
telle requête, et dans ce cas la relation n’est pas créée.} Chaque utilisateur
dispose d’une page d’accueil qui lui présente le flux des publications de ses
\q{amis}. Il peut aussi ajouter une publication sur le \I{mur} d’un de ses
\I{amis}. Ces publications sur le profil d’un autre sont visibles par les
\I{amis} du receveur, et sont une forme de messages publics envoyés d’un
individu à l’autre en ayant conscience que les autres y ont accès
\citep{Walther2008}.

Toutes ces publications peuvent être commentées par les \q{amis} du
propriétaire du \I{mur} sur lequel elles sont écrites. Ces commentaires sont
ordonnés par date et sont complétés par une marque d’approbation non datée que
\fb{} appelle le \I{like} (\q{j’aime} dans la version française). \q{Aimer} une
publication consiste simplement à cliquer sur un bouton correspondant, ce qui
ajoute le nom de l’utilisateur à la liste de tous les \q{amis} qui \I{aiment}
ladite publication. La liste de ces utilisateurs est visible sur une
publication. Cette fonctionnalité est présente non seulement sur les
publications, mais également sur les commentaires et sur les \q{pages}, des
profils particuliers dédiés à une marque ou un produit. Un utilisateur peut
ainsi \I{aimer} des bien culturels, marques diverses, séries télévisées,
personnages célèbres, etc. Ces pages sont listées dans son profil, classées par
catégorie, et visibles par tous les \I{amis} de l’utilisateur.

Enfin, ces différents types de publications peuvent être \I{partagées} par un
\I{ami} de l’auteur original, ce qui lui permet de faire profiter ses propres
\I{amis} de la publication originale. Ces republications (\I{shares}) sont
identifiées comme telles et non comme des publications originales.

\begin{figure}[ht]
    \begin{center}
        \includegraphics[width=.5\textwidth]{figures/images/statut-fb.png}
    \end{center}
    \caption{\label{statut-fb}Une publication sur \fb}
\end{figure}

La figure~\ref{statut-fb} montre un exemple de publication. Celle-ci a été
écrite le 10 décembre 2010 par Bastien Patrick sur son \I{mur}. Elle a été
\I{aimée} par André Dupont, David Alexandre et deux autres \I{amis}. Elle a été
commentée ce même jour à 13h18, 13h55 et 13h57, d’abord par deux amies de
Bastien, puis par Bastien lui-même. Le commentaire de Bastien a été \I{aimé}
par un de ses amis.

Notons que cette relation d’\q{amitié} entre deux personnes sur \fb{} n’est pas
nécessairement l’expression d’une véritable amitié et qu’il est courant d’avoir
des profils avec plusieurs centaines voire milliers d’\q{amis}
\citep{Hogan2010}, bien au delà du nombre maximal de relations maintenables
de 150 suggéré par Dunbar \citep{Dunbar2012,Roberts2011}.

\subsection{Egos et alters}

On introduit les notions d’\q{ego} et d’\q{alter} déjà mentionnées dans le
début de ce document. Les données récoltées dans le cadre de l’enquête
\I{Algopol} se concentrent sur l’individu et son cercle d’\I{amis}. Les
données des différents enquêtés sont cloisonnées de telle sorte qu’il n’est pas
possible d’étudier le corpus comme un graphe d’individus liés plus ou moins
fortement entre eux. Dans chaque cas étudié, l’individu sujet est l’\q{ego},
tandis que tous les individus qui interviennent dans ses interactions sont
\I{ses} \q{alters}. L’\I{ego} est donc l’individu central tandis que les
\I{alters} sont des personnages secondaires mais néanmoins essentiels car il
n’y aurait pas d’interactions sans eux. Nous disposons uniquement des
interactions entre l’\I{ego} et ses \I{alters}, et pas de celles ayant lieu
entre ces \I{alters}.

\subsection{Un corpus, deux ensembles}

Enfin, on désignera par \q{panel \sc csa} l’ensemble des enquêtés issus du
panel représentatif constitué par l’institut {\sc csa}, et par \q{hors-panel}
l’ensemble des enquêtés qui ont utilisé l’application en dehors de ce cadre-là
(avec une diffusion en \q{boule de neige}).
