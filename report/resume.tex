\documentclass[]{article}

\usepackage[francais]{babel}

\usepackage{amssymb,amsmath}
\usepackage{ifxetex,ifluatex}
\usepackage{epsfig,graphics,graphicx}
% \usepackage{pstricks,pst-node,pst-tree}
\ifxetex
  \usepackage{fontspec,xltxtra,xunicode}
  \defaultfontfeatures{Mapping=tex-text,Scale=MatchLowercase}
\else
  \ifluatex
    \usepackage{fontspec}
    \defaultfontfeatures{Mapping=tex-text,Scale=MatchLowercase}
  \else
    \usepackage[utf8]{inputenc}
  \fi
\fi

\usepackage[unicode=true]{hyperref}
\usepackage[babel=true]{csquotes}
\hypersetup{breaklinks=true, pdfborder={0 0 0}}
\setlength{\parindent}{0pt}
\setlength{\parskip}{6pt plus 2pt minus 1pt}
\setlength{\emergencystretch}{3em}  % prevents overfull lines
% \setcounter{secnumdepth}{0}

\usepackage[T1]{fontenc}
\usepackage{lmodern}

\newcommand{\I}{\textit}
\newcommand{\q}{\enquote}
\newcommand{\hash}{\texttt}

\def\fb{Facebook}
\def\twt{Twitter}
\def\humanum{Huma-Num}
\def\api{\I {\sc api}}

\def\csa{\I{\sc csa}}
\def\nocsa{\I{hors-{\sc csa}}}

\def\pnas{{\sc pnas}}

\author{Baptiste Fontaine}
\date{25 juin 2014}

\begin{document}
\thispagestyle{empty}
\begin{center}
    \Large\textbf{Analyse de la sociabilité sur Facebook : Extraction d’indicateurs}\\
    ~\\ % hacky spacing
\end{center}

L’analyse des processus de circulation de l’information est un domaine connu en
sciences sociales, qui a vu l’apparition d’une nouvelle dimension avec
l’arrivée des réseaux sociaux en ligne tels que \fb{} ou \twt{}.
Ces sites ont comme point commun de permettre à leurs utilisateurs de produire
et partager du contenu avec d’autres utilisateurs, et de fournir un lien entre
eux. L’\I{amitié} sur \fb{} lie deux \I{amis} entre eux qui ont de ce fait
chacun accès au contenu de l’autre, et l’\I{abonnement} sur \twt{} permet à
un utilisateur de \I{suivre} le contenu produit par la cible de l’abonnement.

L’arrivée de ces plateforme rend un grand volume de données disponibles pour
les recherches, qu’il n’est pas évident d’exploiter. Ce {\sc tre}, réalisé dans
le cadre du projet \I{Algopol} qui regroupe des chercheurs de plusieurs
horizons différents autour de la problématique de circulation de l’information
entre un utilisateur et ses \I{amis} sur \fb{}, a consisté en la recherche et
la définition d’indicateurs sur des données anonymisées issues de profils
d’utilisateurs \fb{} afin de mieux les comprendre et fournir des bases de
recherche pour les membres du projet.

Ces indicateurs se regroupent en trois familles, comprenant les indicateurs
déjà présents dans les données, les indicateurs calculés comme le nombre
d’\I{amis} ou de contenus publiés, et les indicateurs temporels qui se basent
sur les différents motifs que l’on peut observer en suivant l’évolution du
profil d’un utilisateur \fb{} à travers le temps. On remarque ainsi des motifs
qui se répètent plus ou moins, et permettent de classer les utilisateurs en
fonction de leurs usages de la plateforme. Ainsi, certains ont une activité
très régulière tandis que d’autres ne publient du contenu que rarement, par
\q{pics} d’activité. Il faut pouvoir repérer ces motifs puis automatiser leur
reconnaissance de façon pertinente.

En parallèle de la recherche de ces indicateurs, j’ai développé une application
en ligne de présentation de ceux-ci, tant au niveau global (agrégé) que pour
chacun des 11~000 utilisateurs du corpus dont on dispose. Cette application
est une plateforme centrale dans ce projet, puisqu’elle permet de mieux
comprendre les données du projet, qui ne sont pas disponibles dans un format
lisible par un humain, et en trop grand volume pour être parcourues à la main.

Enfin, j’ai étudié un article de J. Saram\"aki, dans lequel il décrit une
expérience montrant la constance du modèle de communication de lycéens puis
étudiants, indépendamment de la variation de leur cercle d’amis. J’ai pu
reproduire des calculs présentés dans l’article sur les données associées, et
montrer qu’il est possible de reproduire cette méthodologie sur les données
\fb{} que nous avons ici.

Les indicateurs extraits servent de base à des chercheurs pour plusieurs pistes
de recherche différentes, et continueront à servir une fois ce {\sc tre}
terminé.

\end{document}
