%!TEX root = rapport.tex

\section{Perspectives}

Ce {\sc tre} terminé, plusieurs pistes de recherche restent ouvertes. En
particulier, les indicateurs temporels sont intéressants car complexes, et
peuvent apporter beaucoup car il est rare d’avoir beaucoup de données de ce
type à la fois relativement fiables statistiquement et couvrant plusieurs
années. L’application Web va de ce fait évoluer pour permettre la consultation
des histogrammes et indicateurs temporels directement depuis la page d’un
enquêté.

Une seconde piste intéressante est l’étude des signatures sociales, afin de
reproduire l’expérience de Saram\"aki avec des interactions sur \fb{}.
L’application réalisée pendant ce {\sc tre} permet de cibler précisément les
enquêtés pour une telle expérience, ce qui permet de s’assurer d’avoir toutes
les données nécessaires alors que pour l’étude de Saram\"aki une partie
des étudiants ne sont pas allés jusqu’au bout de l’expérience (6 sur 30). Il
n’y a pas de telle incertitude ici, puisque toutes les données sont déjà
disponibles.
