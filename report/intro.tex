%!TEX root = rapport.tex

\section{Introduction}

L’analyse des processus de circulation de l’information n’est pas un domaine
nouveau en sciences sociales, mais l’arrivée des sites de réseaux sociaux en
ligne tels que \fb{} ou \twt{} ajoute une nouvelle dimension aux
recherches. Les grands volumes de données brutes disponibles nécessitent la
collaboration des sociologues et informaticiens afin que les premiers puissent
étudier l’information exploitable extraite par les seconds. Il y a un besoin
d’\emph{organisation} de l’information et d’indicateurs permettant de trier les
données recueillies pour mieux les exploiter et les comprendre.

Nous nous intéressons ici aux réseaux ego-centrés représentés sous forme de
graphes, c’est-à-dire des réseaux d’interactions où l’on se concentre sur un
individu (l’\I{ego}) et ses interactions avec d’autres individus (ses
\I{alters}). Ces réseaux permettent de mettre en évidence les processus de
recommandation personnelle et d’étudier les interactions d’une personne en
particulier plutôt que sa place dans un graphe, avec différents modèles de
communication observables \citep{Raux2011}.

Le réseau social en ligne \fb{} se prête bien à ces recherches, car,
contrairement à des réseaux sociaux plus \q{publics} comme \twt{}, où les
publications sont visibles par défaut, \fb{} se concentre sur les \I{amis}
de l’utilisateur, et les publications sont de ce fait limitées aux \I{amis}
tels que définis par le site\footnote{voir section~\ref{sec:terminologie} pour
la terminologie}.

Ce {\sc tre} a été un travail, en collaboration avec notamment Irène Bastard,
doctorante en sociologie sous la direction de Dominique Cardon, de recherche et
définition d’indicateurs sur des profils issus de \fb{}, et d’extraction
et de présentation de ces indicateurs. Il donne également des perspectives sur
l’exploitation de ceux-ci, en particulier à travers l’étude des signatures
sociales.
