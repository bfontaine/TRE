%!TEX root = rapport.tex

\section{Indicateurs}

Les indicateurs définis avec Irène Bastard ont été extraits des données de la
même façon pour chacun des deux ensembles qui forment le corpus des enquêtés,
mais ont ensuite été présentés de façon séparée.

Ils se divisent en trois grandes familles : les indicateurs \I{présents},
\I{calculés} et \I{temporels}.

\subsection{Procédé d’extraction}

L’extraction s’est faite directement sur chacune des archives, qui n’ont pas
été décompressées entièrement mais plutôt à la volée~: le programme
d’extraction parcourt une archive pas à pas, et à chaque répertoire d’ego
rencontré, le décompresse dans un répertoire temporaire, appelle les fonctions
d’extraction de chacun des indicateurs sur les fichiers extraits, puis supprime
ce répertoire et passe au suivant. Ce procédé, bien que plus lent qu’une
décompression complète, permet de limiter l’espace utilisé.\footnote{les
archives compressées pour le \csa{} et \nocsa{} occupent respectivement 1.1Go et
29Go.}

Plusieurs extractions ont été nécessaires pour s’adapter aux changements
d’indicateurs ainsi qu’à l’ajout de nouveaux durant toute la période du {\sc
tre}.

Les indicateurs extraits sont stockés dans une base de données permettant un
accès rapide et l’exécution de requêtes.

\subsection{Indicateurs présents}

Les indicateurs présents sont directement extractables des données et
concernent principalement les valeurs non-numériques, telles que le sexe, le
code postal et la catégorie socio-professionnelle. Certaines informations se
recoupent, par exemple le sexe de l’enquêté est présent dans les données du
formulaire de l’application \I{Algopol} et dans le profil \fb{}, bien
que pas toujours renseigné. On garde dans ce cas les deux informations, car
elles peuvent parfois différer.\footnote{Certains enquêtés du corpus ont
renseigné un sexe différent dans l’application \I{Algopol} et dans leur profil
\fb{}.} L’âge de l’enquêté est également fourni via le formulaire de
l’application.

Ces indicateurs sont ensuite normalisés, les données du formulaire étant en
français alors que celles de l’\api{} de \fb{} étant en anglais.

\begin{figure}[ht]
    \begin{center}
        \includegraphics[width=.75\textwidth]{figures/ages/csa.png}
    \end{center}
    \caption{\label{ages-csa}Répartition des âges sur l’ensemble \csa{}}
\end{figure}

\begin{figure}[ht]
    \begin{center}
        \includegraphics[width=.75\textwidth]{figures/ages/all.png}
    \end{center}
    \caption{\label{ages-all}Répartition des âges sur l’ensemble \nocsa{}}
\end{figure}

Les figures \ref{ages-csa} et \ref{ages-all} montrent la répartition des âges
des deux ensembles en fonction du sexe de l’enquêté.\footnote{C’est le sexe
donné sur le formulaire de l’application, plus fiable que celui du profil
\fb{}.} Elles illustrent bien la différence de répartition des âges et
des sexes entre les deux ensembles, avec un ensemble \csa{} relativement
homogène (même si le pic féminin à 30--34 ans peut surprendre) et un ensemble
\nocsa{} plutôt jeune et masculin.

\subsection{Indicateurs calculés}

Une large partie des indicateurs nécessitent d’être calculés pour être
produits. On y trouve les indicateurs quantitatifs comme le nombre d’\I{amis}, le
nombre de publications, etc.

\subsubsection{Quantités}

La plupart des indicateurs quantitatifs sont simplement des longueurs de listes
fournies par l’\api{} de \fb{}, parfois filtrées pour avoir des indicateurs
plus précis (par exemple le nombre d’\I{ami(e)s} de chaque sexe) ou des
nombres détaillés par type de publication.

\begin{figure}[ht]
    \begin{center}
        \begin{tabular}{l*{5}{r}}
Indicateur             &  min & médiane & moyenne & max     \\
\hline
\I{Amis}               &  0   & 266     & 364     & 4~905   \\
\I{Amies} femmes       &  0   & 115     & 158     & 2~963   \\
\I{Amis} hommes        &  0   & 140     & 197     & 4~088   \\
Publications           &  0   & 2~510   & 3~571   & 130~751 \\
\I{Likes} reçus        &  0   & 480     & 1~003   & 59~600  \\
Commentaires reçus     &  0   & 649     & 1~300   & 72~624  \\
Publications partagées &  0   & 108     & 405     & 34~127  \\
\I{Pages aimées}       &  0   & 20      & 45      & 1~527   \\
Photos                 &  0   & 38      & 105     & 16~901  \\
Liens partagés         &  0   & 317     & 670     & 50~496  \\
        \end{tabular}
    \end{center}
    \caption{\label{indicators}Indicateurs quantitatifs sur l’ensemble \nocsa{}}
\end{figure}

La figure \ref{indicators} montre une partie des indicateurs calculés avec leurs
valeurs minimales, maximales, médianes et moyennes pour l’ensemble
\nocsa{}.\footnote{Cet ensemble représente 11~441 valeurs. Le sexe des amis se
base sur le champ correspondant dans le profil \fb{} de chacun d’entre eux,
lorsqu’il est renseigné.}

\subsubsection{Liens}

En plus du nombre total de liens partagés, ceux-ci sont également extraits puis
triés en fonction du domaine de leur {\sc url} et ordonnés par nombre de
partages décroissants. Cela permet de voir quels sont les domaines les plus
partagés par un enquêté, et après qualification de ces domaines, de catégoriser
les enquêtés en fonction des liens partagés.

\subsection{Indicateurs temporels}

Les indicateurs temporels profitent du fait que l’on dispose de données sur
plusieurs années (respectivement $4.2$ et 5 ans en moyenne sur les ensembles
\csa{} et \nocsa{}).

Il est nécessaire, pour trouver des indicateurs pertinents, de générer un grand
nombre d’histogrammes montrant l’évolution d’une métrique voulue, de façon à
pouvoir les observer et trouver des motifs à traduire en indicateurs. J’ai
inclus dans ce rapport deux exemples de mosaïques de neuf histogrammes
chacunes, mais le script qui les génère est paramétrable et a été notamment
utilisé pour des mosaïques de $8\times20$ histogrammes.

\begin{figure}[ht]
    \label{indicators:pubs}
    \begin{center}
        \includegraphics[width=.85\textwidth]{figures/plots/time/pubs-5y-3x3.pdf}
    \end{center}
    \caption{Publications mensuelles pour quelques enquêtés pendant 5 ans}
\end{figure}

La figure~\ref{indicators:pubs} montre la quantité de publications mensuelles
pour neuf enquêtés sélectionnés dans l’ensemble \csa{}, sur cinq ans à compter
de leur première publication sur \fb{}. Chaque histogramme donne les premiers
caractères de l’identifiant de l’enquêté et le nombre maximal de publications
mensuelles sur ces cinq ans. Les valeurs en abscisses sont les mois, et les
valeurs en ordonnées ne sont pas données car nous nous intéressons plus aux
motifs d’histogrammes qu’à leurs valeurs exactes.

La première ligne montre des exemples du motif de \q{réveil tardif}, où les
enquêtés concernés se sont inscrits sur la plateforme, puis ont \q{oublié} leur
compte, avec pas (ou très peu) d’activité pendant deux à quatre ans, puis d’un
seul coup une forte activité. Le second enquêté de la ligne (\hash{2ba7a2f6}),
par exemple, a eu une activité négligeable pendant près de quatre ans avant de
commencer une activité intense (plus d’une centaine de publications mensuelles)
jusqu’à la fin de la période observée.

% TODO parler du script de détection du réveil tardif, en expliquant en gros ce
% qu'il fait et son taux de réussite

La deuxième ligne montre des enquêtés avec une activité moyenne, et l’on
distingue bien deux pics d’activité sur le troisième d’entre eux
(\hash{1e144f47}). Ce motif d’activité avec plusieurs pics séparés par des
périodes d’activité modérée se retrouve dans une partie des enquêtés étudiés
pour ces indicateurs, très peu d’entre eux ont une réelle activité régulière
sur une période couvrant plusieurs années.

Enfin, les deux premiers histogrammes de la troisième ligne montrent un autre
motif assez courant où l’on observe une activité en augmentation régulière,
puis une baisse plus ou moins régulière selon les enquêtés, sans reprise
d’activité ensuite. Le troisième histogramme de la ligne illustre un motif
similaire où l’enquêté est très actif dès le début, puis diminue son activité
progressivement jusqu’à un seuil très faible comparé à son activité originelle.

\begin{figure}[ht]
    \label{indicators:friends}
    \begin{center}
        \includegraphics[width=.85\textwidth]{figures/plots/time/friends-5y-3x3.pdf}
    \end{center}
    \caption{Nouveaux \I{amis} mensuels pour quelques enquêtés pendant 5 ans}
\end{figure}

La figure~\ref{indicators:friends} a été réalisée de la même façon que la
figure~\ref{indicators:pubs}, mais représente le nombre de \emph{nouveaux}
\I{amis} mensuels.

Ici, la première ligne montre trois exemples d’enquêtés avec peu de nouveaux
\I{amis} mensuels. Le premier est un homme agé de 23 ans au moment de
l’enquête, avec 50 amis après près de 6 ans d’activité modérée (470
publications), alors que la troisième est une femme avec dix fois plus de
publications sur la même période et 95 amis. Le premier a reçu à peine 20
\I{likes} sur ses publications, alors que la seconde en a reçu plus de $4~000$.
La croissance du nombre de nouvelles relations n’est donc pas toujours liée à
l’activité sur la plateforme, observée en terme de publications.

La ligne du milieu montre des enquêtés plus actifs, avec les deuxièmes et
troisièmes qui montrent un motif relativement courant, où l’enquêté ajoute
toutes les personnes qu’il connait sur la plateforme puis n’ajoute quasiment
plus personne après ce pic initial.

La troisième ligne montre des enquêtés actifs nettement plus régulièrement sur
ce point. On remarquera une très nette pause d’un an sur le dernier enquêté
(\hash{9d043fa8}), un homme de 60 ans au moment de l’enquête, avec $1~080$ amis
et inscrit sur la plateforme depuis plus de 6 ans.


Une fois différents motifs remarqués, il est nécessaire d’automatiser le
processus de reconnaissance pour l’exécuter sur l’ensemble du corpus. Cette
automatisation prend du temps car il est nécessaire de tester plusieurs
hypothèses pour tenter de reconnaître un motif donné de façon simple tout en
évitant les faux négatifs et surtout les faux positifs. La
figure~\ref{classified-pubs} montre un exemple de résultats d’une fonction que
j’ai écrite pour reconnaître le motif \q{réveil tardif}. Les histogrammes
colorés en orange sont rejetés par la fonction, tandis que ceux colorés en vert
sont reconnus comme suivant ce motif.

\begin{figure}[ht]
    \label{classified-pubs}
    \begin{center}
        \includegraphics[width=.85\textwidth]{figures/plots/time/classified-pubs-5y-4x2.pdf}
    \end{center}
    \caption{Classification d’histogrammes de publications}
\end{figure}

La fonction prend la liste des valeurs d’un histogramme, et a deux paramètres.
Le premier, $m$, est entier et représente le nombre maximal de publications
mensuelle qu’un enquêté peut avoir avant d’être considéré actif. Le second,
$t$, est réel, avec $t\in[0-1]$, et représente la part maximale d’activité
possible pendant la période d’\q{endormissement} par rapport à l’activité
totale pendant la période d’observation. De façon plus formelle, on définit la
fonction $activity$~:

$$activity(a, b) = \sum_{i=a}^{b} v_i$$

Cette fonction est ensuite utilisée par la fonction $f$ de reconnaissance du
motif~:

% spacing/alignement à améliorer ?
$$f_{m,t}((v_0, v_1, \ldots, v_n)) = \left\{\begin{array}{ll}
    1 & \mbox{si } \exists k\ge12 \mbox{ tel que } (\ast) \\
    0 & \mbox{sinon}
\end{array}\right.$$

$$(\ast)\;\left\{\begin{array}{l}
        v_k \ge m \\
        \forall i \in [0-k[\;\; v_i < m \\
        activity(0, k-1)/activity(0, n) < t
\end{array}\right.$$\\

On ne prend que les valeurs de $k$ supérieures ou égales à 12 car on souhaite
n’avoir que les enquêtés qui ont été inactifs pendant une période d’au moins un
an (12 mois), mais cette valeur pourrait être modifiée. Les valeurs utilisées
pour la production des exemples de la figure~\ref{classified-pubs} sont $m=5$
et $t=0.05$.

Le dernier histogramme de la première ligne semble correspondre à notre motif,
mais l’enquêté a eu une activité régulière pendant toute la période étudiée,
contrairement aux histogrammes en vert où il y a clairement une période
d’inactivité au début.

Le dernier histogramme de la seconde ligne semble être un faux négatif, mais
cette impression est dûe à l’échelle linéaire utilisée pour les ordonnées. Le
graphique est écrasé par les valeurs maximum, cet enquêté ayant ajouté plus de
$1 200$ publications en un mois peu avant la fin de la période d’observation.

\subsection{Présentation}

Afin d’avoir une présentation claire de ces indicateurs, j’ai, dans le cadre de
ce travail, réalisé une application Web en Python permettant de consulter les
indicateurs d’un enquêté (figure~\ref{app:ego}) ainsi que d’avoir une vue
aggrégée de toutes les valeurs pour chacun des ensembles
(figure~\ref{app:home}), de façon similaire au tableau de la
figure~\ref{indicators} mais plus complète.

\setlength{\abovecaptionskip}{0pt}
\begin{figure}[ht]
    \begin{center}
        \includegraphics[width=.75\textwidth]{figures/images/app/home.png}
    \end{center}
    \caption{\label{app:home}Page d’accueil}
\end{figure}

\begin{figure}[ht]
    \begin{center}
        \includegraphics[width=.75\textwidth]{figures/images/app/ego.png}
    \end{center}
    \caption{\label{app:ego}Page d’un enquêté}
\end{figure}
\setlength{\abovecaptionskip}{10pt}

Cette application constitue la plateforme centrale pour le traitement des
données de l’enquête \I{Algopol}, et est donc utilisée par plusieurs membres du
projet sur plusieurs pistes de recherche différentes.

\setlength{\abovecaptionskip}{0pt}
\begin{figure}[ht]
    \begin{center}
        \includegraphics[width=.75\textwidth]{figures/images/app/links.png}
    \end{center}
    \caption{\label{app:links}Liens partagés par un enquêté}
\end{figure}

\begin{figure}[ht]
    \begin{center}
        \includegraphics[width=.75\textwidth]{figures/images/app/search.png}
    \end{center}
    \caption{\label{app:search}Recherche par identifiant}
\end{figure}
\setlength{\abovecaptionskip}{10pt}

L’application permet également de voir les liens partagés par un enquêté
(figure~\ref{app:links}), de prendre des notes privées sur chacun d’entre eux
et d’en marquer pour les retrouver plus tard, ainsi que de chercher des
enquêtés par leur identifiant (figure~\ref{app:search}).

Elle divisée en deux parties similaires, une pour chacun des ensembles qui
forment le corpus des enquêtés. Chaque partie permet d’exporter tous les
indicateurs de l’ensemble correspondant au format CSV, exploitable par un
tableur.

J’ai apporté une attention particulière à de nombreux petits détails permettant
de faciliter le travail des utilisateurs de l’application. Par exemple, les
identifiants (\I{hashs}) des enquêtés étant suffisamment aléatoires, il
n’existe pas deux enquêtés avec un identifiant commençant par un même préfixe
de 8 caractères. C’est pourquoi l’application raccourcit ceux-ci par défaut
pour prendre moins d’espace sur la page.

Pour raccourcir le temps d’extraction des indicateurs, les liens partagés par
les enquêtés ne sont pas extraits en même temps que le reste, mais au moment où
la page d’un enquêté est visitée : un appel asynchrone au serveur par le
navigateur avec \I{\sc ajax}\footnote{\q{Asynchronous JavaScript And {\sc
Xml}}, un moyen en JavaScript de communiquer au serveur sans bloquer les
interactions sur la page.} est réalisé, et le serveur extrait les liens de
l’enquêté et les insert dans la base de données s’ils ne sont pas déjà présents
puis les envoit au navigateur, qui les injecte dynamiquement dans l’onglet
correspondant. Ce procédé, invisible pour l’utilisateur, permet d’avoir un
temps de chargement des pages rapide quelque soit le nombre de liens partagés
par l’enquêté.

Enfin, j’ai défini l’architecture et conçu l’application de façon modulaire
pour rendre le plus simple possible l’ajout de nouveaux indicateurs. J’ai
travaillé avec Gérald Foliot, ingénieur responsable d’\humanum{}, pour la mise
en ligne de l’application sur leurs serveurs.

Développée pour la continuité, l’application sera progressivement reprise par
un autre étudiant à la fin du {\sc tre}, qui continuera son développement.
