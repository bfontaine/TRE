%!TEX root = rapport.tex

\section{Signatures sociales}

En plus de l’extraction de ces indicateurs, ce corpus permet l’étude des
\q{signatures sociales}. Dans un article publié dans \pnas{} en 2014
\citep{Saramaki2014}, J. Saram\"aki décrit une expérience réalisée avec une
vingtaine de lycéens en dernière année basée sur des appels téléphoniques.
L’expérience s’est déroulée sur dix-huit mois, pendant laquelle le réseau
social proche des étudiants a connu de forts changements puisqu’ils ont quitté
le lycée pour aller à l’université \citep{Paul2001} ou pour une minorité
d’entre eux rejoindre le monde professionnel. Les métadonnées des appels des
lycéens sortants (puis étudiants) vers leurs proches ont été enregistrées
pendant toute la période, puis utilisées pour construire une \I{signature
sociale} pour chaque enquêté sur trois périodes de six mois. Une signature est
une courbe avec en abscisse le rang de chacun des \I{alters} de l’enquêtés
ordonnés par nombre d’appels, et en ordonnées la fraction de ses appels par
rapport à tous les appels de l’enquêté sur la période.

\begin{figure}[ht]
    \begin{center}
        \includegraphics[width=.75\textwidth]{figures/saramaki.pdf}
    \end{center}
    \caption{\label{fig:saramaki}Courbes de Sar\"amaki}
\end{figure}

Un exemple de signatures est donné en figure~\ref{fig:saramaki}, extraite de
l’article. Sur cette figure, les trois signatures du haut correspondent à un
enquêté pour trois périodes de six mois, et les trois du bas correspondent à un
autre enquêté pendant ces mêmes périodes. Les couleurs des points correspondent
au semestre pendant lequel l’\I{alter} correspondant a été vu pour la première
fois.

Dans cet article, Saram\"aki souhaite montrer que les signatures sociales
persistent à travers le temps pour un individu donné, et ce même avec de forts
changements dans le réseau social.

Afin de mieux comprendre la méthodologie, j’ai reproduit les signatures
sociales de quelques enquêtés de cette étude, en utilisant les données de
l’article.

\setlength{\abovecaptionskip}{0pt}
\begin{figure}[ht]
    \begin{center}
    \begin{minipage}[t]{.4\linewidth}
        \begin{center}
    \includegraphics[width=1\textwidth]{figures/plots/saramaki/sigs/e14-sig.pdf}
        \end{center}
    \end{minipage}
    \begin{minipage}[t]{.4\linewidth}
        \begin{center}
    \includegraphics[width=1\textwidth]{figures/plots/saramaki/sigs/e24-sig.pdf}
        \end{center}
    \end{minipage}
    \end{center}
    \caption{\label{fig:saramaki-repro}Reproduction de courbes de Saram\"aki}
\end{figure}
\setlength{\abovecaptionskip}{10pt}

La figure~\ref{fig:saramaki-repro} montre les courbes pour deux enquêtés sur
les six premiers mois de l’étude.

Cette étude pourrait être appliquée de façon similaire sur les données du
corpus dont l’on dispose. Les indicateurs présentés précédemment facilitent la
sélection d’enquêtés pour ce type d’études, et la principale difficulté réside
plus dans le choix des métriques à utiliser que dans la quantité des données
disponibles. Ainsi, Saram\"aki n’a utilisé que le nombre d’appels téléphoniques
(et non leur durée), tandis que nous disposons de plus d’interactions comme
les commentaires, \I{likes}, et publications des \I{alters} sur le \I{mur} de
l’enquêté.

\setlength{\abovecaptionskip}{0pt}
\begin{figure}[ht]
    \begin{center}
    \begin{minipage}[t]{.4\linewidth}
        \begin{center}
    \includegraphics[width=1\textwidth]{figures/plots/saramaki/sigs/001b846d588984fe534083587b094f813fc02e76-sig.pdf}
        \end{center}
    \end{minipage}
    \begin{minipage}[t]{.4\linewidth}
        \begin{center}
    \includegraphics[width=1\textwidth]{figures/plots/saramaki/sigs/00823e9d866857f6863e600dbe6218232b5f1371-sig.pdf}
        \end{center}
    \end{minipage}
    \end{center}
    \caption{\label{fig:saramaki-fb}Signatures sociales sur \fb{}}
\end{figure}
\setlength{\abovecaptionskip}{10pt}

À titre d’exemple, la figure~\ref{fig:saramaki-fb} montre deux signatures pour
des enquêtés de l’ensemble \nocsa. La métrique retenue ici est le nombre de
publications postées par des \I{alters} sur le profil de l’enquêté. C’est donc
plus une signature passive qu’active qui est présentée ici, puisqu’on compte
les interactions provenant des \I{alters} uniquement. Nous ne disposons pas des
données relatives aux publications de chaque enquêté sur les profils d’autres
utilisateurs de la plateforme. Le script utilisé pour générer ces courbes est
générique et peut donc s’adapter à tout type de métrique voulue.
