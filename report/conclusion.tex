%!TEX root = rapport.tex

\section{Conclusion}

Ce {\sc tre}, qui s’inscrit dans un projet plus vaste et plus long que
celui-ci a été l’occasion de collaborer avec plusieurs personnes, qu’elles
soient issues du domaine informatique (Christophe Prieur, Stéphane Raux) ou
des sciences sociales (Irène Bastard). Il a débuté peu de temps avant la
livraison des données de l’enquête, ce qui a permis d’étudier l’article de
Saram\"aki en attendant puis de développer un prototype d’application sur les
extraits des données fournis par Stéphane Raux pour être opérationnel le plus
rapidement possible. Les données ont été rendues disponibles début avril,
l’application a été mise en ligne le même jour et deux jours plus tard comptait
une vingtaine d’indicateurs extraits des archives. Ces premiers indicateurs ont
permis d’avoir un aperçu des pratiques du corpus, et de confirmer ou infirmer
les hypothèses de chacun. La collaboration interdisciplinaire a été au cœur de
ce travail, durant lequel il a fallu comprendre et interpréter les attentes
\I{sociologiques} d’I. Bastard pour les traduire en réponses \I{informatiques}
pertinentes.

Les indicateurs extraits ont permis de mettre en évidence des motifs d’usages
de \fb{} et d’offrir une base essentielle sur laquelle peuvent s’appuyer les
membres du projet \I{Algopol} pour la sélection d’enquêtés et l’exploration
des données disponibles.
