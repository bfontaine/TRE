%!TEX root = rapport.tex

\section{Données}

Les données issues de l’enquête sont anonymisées puis placées sur un serveur
mis à disposition par Huma-Num, \q{Très Grande Infrastructure de Recherche
(TGIR)} du CNRS. Elles sont divisées en deux archives compressées, l’une pour
le \I{panel \sc csa}, l’autre pour le \I{hors-panel}.

Elles suivent toutes les deux le même format, à savoir un répertoire par
enquêté, qui contiennent tous quatre fichiers~:

\begin{itemize}
    \item \texttt{qualify.json} contient les informations renseignées dans le
        questionnaire de l’application \I{Algopol}, comme le code postal et le
        sexe.
    \item \texttt{ego.json} contient les informations extraites du profil de
        l’enquêté, comme la liste (anonymisé) de ses \I{amis} et les \I{pages}
        qu’il a \I{aimé}.
    \item \texttt{friends.jsons} contient l’équivalent de \texttt{ego.json}
        pour chacun des \I{amis} de l’enquêté.
    \item \texttt{statuses.jsons} contient l’ensemble des publications sur le
        profil de l’enquêté, qu’elles soient publiées par lui-même ou par un de
        ses \I{amis}.
\end{itemize}

Excepté le premier, tous les fichiers sont des exports de l’\api{} de
\fb{}. Ces données sont organisées de façon peu claire, et il a donc
fallu expérimenter avec le format et tester plusieurs hypothèses pour
déterminer à quoi correspondait chacun des champs, l’anonymisation et la
compression des données ne facilitant pas la tâche car il manque les champs
correspondant au contenu textuel et les valeurs anonymisées sont toutes dans le
même format : un \I{hash} de 40 caractères, quelle qu’ait été la valeur
d’origine. De plus, les listes, notamment de publications, ne sont pas triées,
on ne peut donc pas se fier à la position d’une publication dans la liste pour
avoir son ancienneté.

Les champs disponibles varient d’enquêté à enquêté car tous n’ont pas
renseigné les mêmes informations, et les fonctionnalités de \fb{} ont
varié dans le temps~; il n’y a pas toujours eu des publications, \I{likes} et
commentaires, certaines fonctionnalités ont disparu depuis leur création,
d'autres ont évolué. De plus, tous les enquêtés ne se sont pas inscrits au même
moment, et de ce fait nous n’avons pas la même période d’activité pour tous.

J’ai pu collaborer avec Stéphane Raux, doctorant chez Linkfluence, qui a
développé l’application de collecte des données, et ainsi pu lui faire des
retours sur le format utilisé pour l’export des données, qui va évoluer pour
prendre en compte ces remarques et les attentes des chercheurs qui utilisent
ces données.

Un tel volume de données rend difficile l’exploration du corpus si l’on ne
dispose pas d’indications permettant d’avoir une idée globale de l’activité des
enquêtés, ou de critères pour les catégoriser. Tous les chercheurs ne sont pas
informaticiens, les sociologues par exemple ne peuvent pas travailler sur des
archives de cette taille sans les outils nécessaires pour les exploiter. Il est
donc important de pouvoir définir des indicateurs permettant d’avoir un aperçu
de l’activité des enquêtés aussi bien au niveau individuel que sur l’ensemble
du corpus.
