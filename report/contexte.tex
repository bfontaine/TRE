%!TEX root = rapport.tex

\section{Contexte}

Ce travail a été réalisé dans le cadre du projet \emph{Algopol}, auquel
participent le {\sc cams} (Centre d’Analyse et de Mathématique Sociale),
laboratoire du {\sc cnrs} et de l'{\sc ehess} (École des Hautes Études en
Sciences Sociales), le {\sc liafa}, Linkfluence (entreprise spécialisée dans la
veille sur le Web) et le laboratoire de sociologie et économie d’Orange Labs,
et qui regroupe des chercheurs de plusieurs disciplines, dont la sociologie,
l’économie et l’informatique.

Pour collecter des données nécessaires aux recherches, \I{Algopol} a mis en
place une application Web, qui, une fois connectée au compte \fb{} d’un
utilisateur qui l’a acceptée, utilise l’\api\footnote{Une \api{}
(\q{\I{Application Programming Interface}}) est une interface permettant à des
programmes d’interagir avec une application, ici \fb{}.} du site pour
récolter toutes les données associées au profil et aux publications de
l’intéressé. En plus de cette collecte automatisée, l’utilisateur doit répondre
à un bref questionnaire personnel comportant notamment son âge, sexe et code
postal, ainsi que la qualification de la relation qu’il entretient avec cinq de
ses \I{amis} (au sens de \fb{}) sur le réseau social et surtout en dehors.
Les utilisateurs peuvent également accepter d’être contactés ultérieurement
pour un entretien avec une sociologue sur leurs usages de la plateforme, ainsi
que de partager un lien vers l’application sous forme de publication sur leur
profil, ce qui permet d’encourager leurs \I{amis} à utiliser l’application.

En échange de cette collecte, l’utilisateur a accès à une représentation
graphique de son réseau d’\I{amis}, avec différents filtres permettants de
mettre en valeur différents types d’interactions sur le graphe, et permet de
visualiser l’évolution de la forme du graphe dans le temps.

La diffusion de l’information concernant l’existence de cette application a
donc suivi un modèle en \q{boule de neige}~: un ensemble relativement restreint
d’utilisateurs a utilisé l’application, puis a fait suivre l’information à un
ensemble un peu plus large, et ainsi de suite. Cet ensemble comptait 11~000
utilisateurs de 18 ans et plus au début du {\sc tre}.

Cette enquête a permis d’obtenir un grand volume de données, mais n’a une
valeur statistique que limitée, puisque dû à la façon dont s’est diffusée
l’information sur l’application, l’ensemble des enquêtés est relativement jeune
(75\% ont 35 ans ou moins) et masculin. Il est courant dans ce type d’enquête
de disposer de beaucoup d’informations sans avoir de corpus représentatif. Pour
compenser cela, \I{Algopol} s’est également associé à l’institut \I{\sc CSA}
pour former un panel d’internautes représentatif de la population française,
qui a ensuite utilisé l’application. Ce second corpus d’enquêtés contient des
informations sur environ 900 utilisateurs de \fb{}, avec une valeur statistique
plus fiable.

Les études de ces deux corpus (les 11~000 enquêtés \q{boule de neige} et les
900 du panel {\sc csa}) ont été séparées de façon à bien distinguer les
observations et résultats obtenus sur l’un par rapport à l’autre, car ceux
obtenus sur le second sont plus généralisables car le corpus est représentatif.

Cette collecte pose des problèmes de protection de la vie privée, ces données
sont donc à disposition des chercheurs sous forme anonymisée. Les champs de
texte sont supprimés, et toutes les données permettant d’identifier un
individu, comme son nom ou sa ville d’origine lorsqu’elle est disponible, sont
\I{hashées}, et ce de façon consistante dans toutes les données liées à un
enquêté. Ce processus donne des données où chaque champ auparavant intelligible
est désormais un \I{hash}, une suite de 40 caractères hexadécimaux. Ces
\I{hashs} sont utilisés pour identifier les enquêtés, car ils sont uniques sur
chacun des corpus~: un identifiant correspond à un et un seul enquêté. La
consistance du \I{hashage} dans les données liées à un enquêté est importante~:
si deux publications sont \I{commentées} par un individu ayant le même
identifiant, c’est que c’est le même dans les deux cas. Cela permet de suivre
l’évolution des interactions entre l’individu étudié (l’\I{ego}) et un autre (un
de ses \I{alters}). En revanche, la méthode de \I{hashage} est différente pour
chaque enquêté, de telle sorte qu’il ne soit pas possible de croiser les
informations d’un enquêté avec un autre. Il n’est par exemple pas possible de
savoir si l’\I{alter} d’un enquêté est lui-même présent en tant qu’enquêté dans
le corpus.
